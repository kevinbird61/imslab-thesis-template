%
% this file is encoded in utf-8
% v2.0 (Apr. 5, 2009)
%%% 每一個附錄 (附錄甲、附錄乙、...) 都要複製此段附錄編排碼做為起頭
%%% 附錄編排碼 begin >>>
\newpage
\chapter*{Appendix A: MATLAB / Octave } % 修改附錄編號與你的附錄名
\phantomsection % for hyperref to register this
\addcontentsline{toc}{chapter}{Appendix A: MATLAB / Octave} %建議此內容應與上行相同
%\setcounter{chapter}{0}  %如果用的是 TeXLive2007 則打開此行以避免錯誤
\setcounter{equation}{0} 
\setcounter{figure}{0} 
\setcounter{footnote}{0} 
\setcounter{section}{0} 
\setcounter{subsection}{0}
\setcounter{subsubsection}{0}
\setcounter{table}{0} 
\renewcommand{\thechapter}{A} % 如果是附錄乙,則內容應為{乙}
%%% <<< 附錄編排碼 end

% 附錄內容開始
% 納入程式源碼
%\lstinputlisting{example_prog_list.m}


\begin{equation}\sum_{k=1}^{n} k = \frac{n(n+1)}{2}\end{equation}

%%% 如果有附錄乙、丙、...,則在此繼續加上「附錄編排」碼
% 每一個附錄會自動以新頁開始